\documentclass[12pt]{article}
\usepackage{graphicx} % Required for inserting images
\usepackage{hyperref}
\usepackage{lineno}
\usepackage{natbib}
% \linenumbers

\title{My first document}
\author{Duccio Rocchini}
% \date{ }

\begin{document}

\maketitle

\tableofcontents

\section{Introduction} \label{sec:intro}
Every year, it's the same and I feel it again
I'm a loser, no chance to win
Leaves start falling, comedown is calling
Loneliness starts sinking in
But I'm one, I am one
And I can see that this is me
And I will be
You'll all see I'm the one
Where do you get those blue, blue jeans?
Faded, patched secret, so tight
Where do you get that walk, oh so lean
Your shoes and your shirts all just right.

\subsection{Philosophy of this thesis}
\noindent Every year, it's the same and I feel it again
I'm a loser, no chance to win
Leaves start falling, comedown is calling
Loneliness starts sinking in
But I'm one, I am one
And I can see that this is me
And I will be
You'll all see I'm the one
Where do you get those blue, blue jeans?
Faded, patched secret, so tight
Where do you get that walk, oh so lean
Your shoes and your shirts all just right.

\subsection{Aim}
Every year, it's the same and I feel it again
I'm a loser, no chance to win
Leaves start falling, comedown is calling
Loneliness starts sinking in
But I'm one, I am one
And I can see that this is me
And I will be
You'll all see I'm the one
Where do you get those blue, blue jeans?
Faded, patched secret, so tight
Where do you get that walk, oh so lean
Your shoes and your shirts all just right.

\section{Methods} \label{sec:methods}
As stated in Section \ref{sec:intro}, in this thesis, I made use of the function related to gravity, as in Equation \ref{eq:newton}.

\begin{equation}
    F = G \times \frac{m_1 \times m_2}{r^2}
    \label{eq:newton}
\end{equation}

The above equation was too simple, and I relied also on Equation \ref{eq:complex}.

\begin{equation}
    F = \sqrt[3]{G \times \frac{\sqrt{m_1 \times \sqrt{m_2}}}{r^2}}
    \label{eq:complex}
\end{equation}

\section{Results and Discussion}

Using the formulas presented in section \ref{sec:methods}, I attained the results shown in Figure \ref{fig:result}.

\begin{figure}
    \centering
    \includegraphics[width=.8\linewidth]{download.png}
    \caption{Waves attained in this thesis, blablabla....}
    \label{fig:result}
\end{figure}

My results are in line with those attained by \citep{lek}. Instead of just gravity I might have used cellular automata according to \citep{codd}.

\begin{thebibliography}{999}

\bibitem[Lek et al.(1996)]{lek}
Lek, S., Delacoste, M., Baran, P., Dimopoulos, I., Lauga, J., \& Aulagnier, S. (1996). Application of neural networks to modelling nonlinear relationships in ecology. Ecological modelling, 90(1), 39-52.

\bibitem[Codd(2014)]{codd}
Codd, E. F. (2014). Cellular automata. Academic press.

\end{thebibliography}

\end{document}
